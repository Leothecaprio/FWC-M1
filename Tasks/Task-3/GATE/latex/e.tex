\documentclass[a4paper,12pt]{article}
\usepackage{graphicx}
\usepackage{amsmath}
\usepackage{geometry}
\usepackage{amsmath, amssymb}
\usepackage{float}
\usepackage{caption}
\usepackage{subcaption}
\usepackage{xcolor}
\usepackage{fancyhdr}
\usepackage{datetime2}
\usepackage{pgfplots}

\definecolor{darkskyblue}{rgb}{0.0, 0.5, 1.0}
\definecolor{skyblue}{RGB}{135, 206, 235}
\usepackage{wrapfig}
\usepackage{circuitikz}

\geometry{a4paper, top=0.7in, left=1in, right=1in, bottom=1in}

\begin{document}

\pagestyle{empty} % Start with empty page style

\thispagestyle{fancy} % Apply fancy style only to the first page
\fancyhf{} % Clear header and footer
\renewcommand{\headrulewidth}{0pt} % Remove header rule

\fancyhead[L]{% Left header
	\includegraphics[width=8cm, height=1.7cm]{i.png} % Adjust dimensions
}
\fancyhead[R]{% Right header
    Name: Sritama Biswas \\
    Batch: COMETFWC011 \\
    Date: 26 march 2025 
}

\vspace{10cm}
\begin{center}
   
    {\LARGE \textbf{\textcolor{darkskyblue}{\\  GATE QUESTION \\ ECE 2009 Q38}}}
\end{center}

\vspace{-1cm} %adjust vertical space

\section*{\textcolor{blue}{\\Question}}
Q38) Refer to the NAND and NOR latches shown in the figure. The inputs $(P_1, P_2)$ for both the latches are first made (0,1) and then, after a few seconds, made (1,1). The corresponding stable outputs $(Q_1, Q_2)$ are:

\vspace{1cm}
\noindent
\resizebox{0.48\textwidth}{!}{
\begin{circuitikz}
    \node[nand port,scale=1] (NAND1) at (0,2) {};
    \node[anchor=east] at (1,2) {$Q_1$};
    \node[anchor=east] at (1,0) {$Q_2$};
    \node[nand port,scale=1] (NAND2) at (0,0) {};
    \draw (NAND1.in 1) -- ++(-1,0) node[anchor=east] {$P_1$};
    \draw (NAND2.in 2) -- ++(-1,0) node[anchor=east] {$P_2$};
    \draw (NAND1.out) -- ++(0,-0.5) -- ($(NAND2.in 1) +(0,0.5)$) -- (NAND2.in 1);
    \draw (NAND2.out) -- ++(0,+0.5) -- ($(NAND1.in 2) +(0,-0.5)$) -- (NAND1.in 2);
\end{circuitikz}
}\hfill
\resizebox{0.48\textwidth}{!}{
\begin{circuitikz}
    \node[nor port,scale=1] (NOR1) at (0,2) {};
    \node[anchor=east] at (1,2) {$Q_1$};
    \node[anchor=east] at (1,0) {$Q_2$};
    \node[nor port,scale=1] (NOR2) at (0,0) {};
    \draw (NOR1.in 1) -- ++(-1,0) node[anchor=east] {$P_1$};
    \draw (NOR2.in 2) -- ++(-1,0) node[anchor=east] {$P_2$};
    \draw (NOR1.out) -- ++(0,-0.5) -- ($(NOR2.in 1) +(0,0.5)$) -- (NOR2.in 1);
    \draw (NOR2.out) -- ++(0,+0.5) -- ($(NOR1.in 2) +(0,-0.5)$) -- (NOR1.in 2);
\end{circuitikz}
}

\vspace{1cm}
\textbf{Options:}
\begin{enumerate}
    \item[(A)] NAND: first (0,1) then (0,1), NOR: first (1,0) then (0,0)
    \item[(B)] NAND: first (1,0) then (1,0), NOR: first (1,0) then (1,0)
    \item[(C)] NAND: first (1,0) then (1,0), NOR: first (1,0) then (0,0)
    \item[(D)] NAND: first (1,0) then (1,1), NOR: first (0,1) then (0,1)
\end{enumerate}

\section*{\textcolor{blue}{Solution}}

\subsection*{1. NAND Latch Analysis}
The NAND latch consists of two cross-coupled NAND gates, with outputs given by:

\[
Q_1 = \overline{P_1 \cdot Q_2}, \quad Q_2 = \overline{P_2 \cdot Q_1}
\]

\textbf{Step 1: Inputs (0,1)}
\[
Q_1 = \overline{0 \cdot Q_2} = \overline{0} = 1
\]
\[
Q_2 = \overline{1 \cdot Q_1} = \overline{1} = 0
\]
\textbf{Output:} \( (Q_1, Q_2) = (1,0) \)

\textbf{Step 2: Inputs (1,1)}
\[
Q_1 = \overline{1 \cdot Q_2} = \overline{1 \cdot 0} = \overline{0} = 1
\]
\[
Q_2 = \overline{1 \cdot Q_1} = \overline{1 \cdot 1} = \overline{1} = 0
\]
\textbf{Output:} \( (Q_1, Q_2) = (1,0) \)

\begin{table}[h]
\centering
\begin{tabular}{|c|c|c|}
\hline
\raisebox{-0.5ex}[0pt][0pt]{A} & \raisebox{-0.5ex}[0pt][0pt]{B} & \raisebox{-0.5ex}[0pt][0pt]{$\overline{A \cdot B}$} \\  % Increase height for this row
\hline
0 & 0 & 1 \\
0 & 1 & 1 \\
1 & 0 & 1 \\
1 & 1 & 0 \\
\hline
\end{tabular}
\caption{Truth Table for NAND Gate}
\end{table}

\vspace{-0.8cm}
\subsection*{2. NOR Latch Analysis}
The NOR latch consists of two cross-coupled NOR gates, with outputs given by:

\[
Q_1 = \overline{P_1 + Q_2}, \quad Q_2 = \overline{P_2 + Q_1}
\]

\textbf{Step 1: Inputs (0,1)}
\[
Q_1 = \overline{0 + Q_2} = \overline{0} = 1
\]
\[
Q_2 = \overline{1 + Q_1} = \overline{1} = 0
\]
\textbf{Output:} \( (Q_1, Q_2) = (1,0) \)

\textbf{Step 2: Inputs (1,1)}
\[
Q_1 = \overline{1 + Q_2} = \overline{1} = 0
\]
\[
Q_2 = \overline{1 + Q_1} = \overline{1} = 0
\]
\textbf{Output:} \( (Q_1, Q_2) = (0,0) \)

\begin{table}[h]
\centering
\begin{tabular}{|c|c|c|}
\hline
\raisebox{-0.5ex}[0pt][0pt]{A} & \raisebox{-0.5ex}[0pt][0pt]{B} & \raisebox{-0.5ex}[0pt][0pt]{$\overline{A + B}$} \\  % Increase height for this row
\hline
0 & 0 & 1 \\
0 & 1 & 0 \\
1 & 0 & 0 \\
1 & 1 & 0 \\
\hline
\end{tabular}
\caption{Truth Table for NOR Gate}
\end{table}

\vspace{-1cm}
\section*{\textcolor{blue}{Final Answer}}
- \textbf{NAND latch:} First (1,0), then (1,0)
- \textbf{NOR latch:} First (1,0), then (0,0)

\textbf{\textcolor{green}{Correct Option:} (C)}
\vspace{1cm}
\textbf{Truth Table for NAND and NOR Latches}

\begin{table}[h]
    \centering
    \renewcommand{\arraystretch}{1.5}
    \begin{tabular}{|c|c|c|c|c|c|}
        \hline
        \textbf{Time Step} & \textbf{Input $(P_1, P_2)$} & \multicolumn{2}{c|}{\textbf{NAND Latch Output $(Q_1, Q_2)$}} & \multicolumn{2}{c|}{\textbf{NOR Latch Output $(Q_1, Q_2)$}} \\
        \hline
        1 & (0,1) & 1 & 0 & 1 & 0 \\
        \hline
        2 & (1,1) & 1 & 0 & 0 & 0 \\
        \hline
    \end{tabular}
    \caption{State Table for NAND and NOR Latches}
    \label{tab:latch_states}
\end{table}


\section*{Timing Diagram for NAND and NOR Latches}

\begin{tikzpicture}
    % Time axis
    \draw[->] (0,0) -- (10,0) node[right] {Time};
    
    % NAND Latch Signals
    \draw (0,3) node[left] {$Q_1$ (NAND)} -- (2,3) -- (2,3) -- (8,3) -- (10,3);
    \draw (0,2) node[left] {$Q_2$ (NAND)} -- (2,2) -- (2,2) -- (8,2) -- (10,2);

    % NOR Latch Signals
    \draw (0,-1) node[left] {$Q_1$ (NOR)} -- (2,-1) -- (2,-2) -- (8,-2) -- (10,-2);
    \draw (0,-2) node[left] {$Q_2$ (NOR)} -- (2,-2) -- (2,-1) -- (8,-3) -- (10,-3);

    % Time markers
    \foreach \x in {2,8} {
        \draw[dashed] (\x,-3.5) -- (\x,4);
    }
    
    % Labels for transitions
    \node[above] at (2,3) {(1,0)};
    \node[above] at (8,3) {(1,0)};
    \node[below] at (2,-1) {(1,0)};
    \node[below] at (8,-1) {(0,0)};

\end{tikzpicture}

\begin{center}
\section*{Graph for NAND and NOR Latches}
    \begin{tikzpicture}[scale=2] 

        % Extend and label x-axis
        \draw[thick,->] (-2,0) -- (2,0) node[right] {\textbf{x}}; 
        
        % Extend and label y-axis
        \draw[thick,->] (0,-2) -- (0,2) node[above] {\textbf{y}}; 

        % Draw the red diagonal line
        \draw[red, thick] (-0.5,1.5) -- (1.5,-0.5);

       

        % Add labels
        \node at (0.1,1.1) {\textcolor{red}{\textbf{(0,1)}}};
        \node at (1.1,-0.1) {\textcolor{red}{\textbf{(1,0)}}};
        \node at (-0.1,-0.1) {\textbf{O}};

    \end{tikzpicture}
\end{center}
\end{document}
